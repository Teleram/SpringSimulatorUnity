\documentclass[compress]{beamer}

\usetheme{Hamburg}

\usepackage[T1]{fontenc}
\usepackage[utf8]{inputenc}

\usepackage{lmodern}

%\usepackage[english]{babel}
\usepackage[ngerman]{babel}

\usepackage{eurosym}
\usepackage{listings}
\usepackage{lstautogobble}
\usepackage{microtype}
\usepackage{textcomp}
\usepackage{units}

\lstset{
	basicstyle=\ttfamily\footnotesize,
	frame=single,
	numbers=left,
	language=C,
	breaklines=true,
	breakatwhitespace=true,
	postbreak=\hbox{$\hookrightarrow$ },
	showstringspaces=false,
	autogobble=true,
	upquote=true,
	tabsize=4,
	captionpos=b,
	morekeywords={int8_t,uint8_t,int16_t,uint16_t,int32_t,uint32_t,int64_t,uint64_t,size_t,ssize_t,off_t,intptr_t,uintptr_t,mode_t}
}

\title{Gaming AI}
\subtitle{Projekt Big Data}
\author{Friedrich Braun, Valentin Krön\\
Betreuer: Eugen Betke, Julian Kunkel}
\institute{Arbeitsbereich Wissenschaftliches Rechnen\\Fachbereich Informatik\\Fakultät für Mathematik, Informatik und Naturwissenschaften\\Universität Hamburg}
\date{2017-03-19}

\titlegraphic{\includegraphics[width=0.75\textwidth]{logo}}

\begin{document}

\begin{frame}
	\titlepage
\end{frame}

\begin{frame}
	\frametitle{Gliederung (Agenda)}

	\tableofcontents[hidesubsections]
\end{frame}

\section{Projektziel}
\subsection{Altes Projektziel}

\begin{frame}
	\frametitle{Altes Projektziel}

	\begin{itemize}
		\item AI-Bot für RTS
		\begin{itemize}
			\item genauer: Spring-Engine
		\end{itemize}
		\item Training via Genetischem Algorithmus
		\item Sprache C oder C++
		\item Orientierung an bestehenden AI's
		\begin{itemize}
			\item gegen bestehende AI's spielen
		\end{itemize}
		\item AI gleichmächtig wie Mensch
		\begin{itemize}
			\item[$\rightarrow$] besserer Spieler
		\end{itemize}
	\end{itemize}
\end{frame}

\begin{frame}
	\frametitle{Warum Änderung des Ziels?}

	\begin{itemize}
		\item Spring ist unübersichtlich aufgebaut
		\begin{itemize}
			\item nicht nachvollziehbare Probleme beim Starten von Spring
			\item nicht modspezifische AI's ließen sich nicht einbinden
		\end{itemize}
		\item Mods mit eigener Lobby
		\begin{itemize}
			\item funktioniert in sich gut
			\item nicht von außen zu erweitern
			\item Mod um AI erweitern (Lua) zu aufwendig
		\end{itemize}
		\item Ziele nicht im Projektzeitraum erreichbar
	\end{itemize}
\end{frame}

\subsection{Neues Projektziel}

\begin{frame}
	\frametitle{Neues Projektziel}

	\begin{itemize}
		\item Proof-of-concept
		\begin{itemize}
			\item AI für RTS
			\item Training mit genetischem Algorithmus
		\end{itemize}
		\item[$\Rightarrow$] Vereinfachtes RTS selber bauen
		\begin{itemize}
			\item Spieleengine Unity [1]
		\end{itemize}
		\item Beibehaltung des direkten Vergleiches mit Mensch
	\end{itemize}
\end{frame}



\section{Spiel}
\subsection{Unity}

\begin{frame}
	\frametitle{Warum Unity?}

	\begin{itemize}
		\item Sollte Spiel bleiben
		\item AI weiterhin gleichmächtig wie Mensch
		\item bekannt
		\begin{itemize}
			\item umfassende Dokumentation
			\item viele (kostenlose) Online-Tutorien
		\end{itemize}
		\item kostenlos für Uniprojekte
		
	\end{itemize}
\end{frame}

\begin{frame}
	\frametitle{Schwierigkeiten mit Unity}

	\begin{itemize}
		\item Unity handhabt Codestrukturen anders
		\begin{itemize}
			\item[$\Rightarrow$] viele Konventionen nicht anwendbar
			\item[$\Rightarrow$] unintuitiv
		\end{itemize}
		\item Multiplayer (Unity Networking)
		\begin{itemize}
			\item jeder Wert muss explizit übertragen werden
			\item Fehler durch verzögerte Übertragung
			\item Objekt-Referenzen nicht übertragbar
			\item netId-Component für uns nicht zugreifbar
		\end{itemize}
	\end{itemize}
\end{frame}

\subsection{C\# und Microsoft Visual Studio}

\begin{frame}
	\frametitle{Warum C\# ? Warum Visual Studio?}

	\begin{itemize}
		\item C\#
		\begin{itemize}
			\item Aktuelle Unityversion unterstüzt C\# und Javascript
			\item bewusst gegen Javascript und für C\# entschieden
			\item C\# relativ nah an bisher Gelerntem
		\end{itemize}
		
		\item Visual Studio
		\begin{itemize}
			\item direkte Verbindung mit Unity
			\item unkompliziert und einfach
			\item keine Suche nach anderer IDE
		\end{itemize}
	\end{itemize}
\end{frame}

\subsection{Neurales Netzwerk}

\begin{frame}
	\frametitle{Neuronales Netzwerk}

	\begin{itemize}
		\item Warum Verwendung eines NN
		\begin{itemize}
			\item Schwarmintelligenz
			\begin{itemize}
				\item viele Teilprobleme geringer Koplexität
				\item Teilprobleme können dieselbe Lösung haben
			\end{itemize}
			\item skalierbare Komplexität
			\item Erweiterbarkeit um neue Funktionalitäten
			\item generieren von Entscheidungen
		\end{itemize}
		\item Umsetzung
		\begin{itemize}
			\item Bibliothek eingebunden
			\begin{itemize}
				\item C\# Neural Network Library [2]
				\item nur Übernahme der Struktur
			\end{itemize}
		\end{itemize}	
	\end{itemize}
\end{frame}

\subsection{Genetischer Algorithmus}

\begin{frame}
	\frametitle{Aufbau}

	\begin{itemize}
		\item selbstgebaut in C\#
		\item Gene sind die Gewichte des NN
		\item Genotyp = Phänotyp		
		\item Fitnesswert
		\begin{itemize}
			\item eigene Einheiten - Gegnereinheiten
		\end{itemize}
		\item zufällige Initialisierung
	\end{itemize}
\end{frame}

\begin{frame}
	\frametitle{Kennfunktionen}

	\begin{itemize}
		\item Rekombination
		\begin{itemize}
			\item random cut
		\end{itemize}
		\item Mutation
		\begin{itemize}
			\item pro Individuum ob mutiert
			\item wenn, dann genau ein Gen zufällig neu
		\end{itemize}
		\item Selektion
		\begin{itemize}
			\item die Fittesten überleben
		\end{itemize}
		
		
	\end{itemize}
\end{frame}

\section{Fazit}
\subsection*{}

\begin{frame}
	\frametitle{Was erreicht?}

	\begin{itemize}
		\item Spiel im lokalen Netzwerk voll funktionsfähig
		\begin{itemize}
			\item genau zwei Spieler
		\end{itemize}
		\item Gleichmächtigkeit erfüllt
		\item unterschiedliche Individuen erzeugen\\
		 unterschiedliche Verhaltensweisen
		\item der genetische Algorithmus funktioniert
		\begin{itemize}
			\item Kommunikation von Gewichten und Fitness funktioniert
		\end{itemize}
	\end{itemize}
\end{frame}

\subsection*{}

\begin{frame}
	\frametitle{Ausblick}

	\begin{itemize}
		\item Spiel startet nicht vollautomatisch
		\begin{itemize}
			\item[$\Rightarrow$] Training nicht vollautomatisch
			\item[$\Rightarrow$] keine Trainingserfolge
		\end{itemize}
		\item Training parallelisieren
		\begin{itemize}
			\item Unity kann nicht im Background laufen
		\end{itemize}
		\item Erweiterung der Komplexität des Spieles
	\end{itemize}
\end{frame}

\section{Quellen}
\subsection*{}

\begin{frame}
	\frametitle{Quellen}

	\bibliographystyle{alpha}
	\bibliography{literatur}
	\begin{enumerate}
		\item \url{https://unity3d.com/}
		\item \url{http://franck.fleurey.free.fr/NeuralNetwork/}
		\item \url{https://msdn.microsoft.com/library/}
	\end{enumerate}
\end{frame}

\end{document}
